\section{Basic}
\subsection{.vimrc}
\lstinputlisting{basic/vimrc}
\subsection{IncStack}
\lstinputlisting{basic/IncStack.cpp}
\subsection{IncStack windows}
\lstinputlisting{basic/IncStack_windows.cpp}
\subsection{random}
\lstinputlisting{basic/random.cpp}
\subsection{time}
\lstinputlisting{basic/time.cpp}

\section{Math}
\subsection{basic}
\lstinputlisting{Math/basic.cpp}
\subsection{MongeDP}
\lstinputlisting{Math/MongeDP.cpp}
\subsection{Chinese Remainder Theorem}
\lstinputlisting{Math/chinese_remainder_theorem.cpp}
\subsection{Discrete Log}
\lstinputlisting{Math/discrete_log.cpp}
\subsection{Discrete Kth root}
\lstinputlisting{Math/discreteKthsqrt.cpp}
\subsection{FFT}
\lstinputlisting{Math/FFT.cpp}
\subsection{FWT}
\lstinputlisting{Math/FWT.cpp}
\subsection{Gauss Lagrange Eisenstein reduced form}
\lstinputlisting{Math/Gauss_Lagrange_Eisenstein_reduced_form.cpp}
\subsection{Lagrange Polynomial}
\lstinputlisting{Math/Lagrange_poly.cpp}
\subsection{Lucas}
\lstinputlisting{Math/Lucas.cpp}
\subsection{Meissel–Lehmer PI}
\lstinputlisting{Math/Meissel–Lehmer_PI.cpp}
\subsection{Miller Rabin with Pollard rho}
\lstinputlisting{Math/Miller_Rabin_with_Pollard_rho.cpp}
\subsection{Mod Mul Group Order}
\lstinputlisting{Math/Mod_Mul_Group_Order.cpp}
\subsection{NTT}
\lstinputlisting{Math/NTT.cpp}
\subsection{Number Theory Functions}
\lstinputlisting{Math/number_theory_functions.cpp}
\subsection{Polynomail root}
\lstinputlisting{Math/rootsOfPoly.cpp}
\subsection{Subset Zeta Transform}
\lstinputlisting{Math/subset_zeta_transform.cpp}

\section{Data Structure}
\subsection{Disjoint Set}
\lstinputlisting{DS/DisjointSet.cpp}
\subsection{Heavy Light Decomposition}
\lstinputlisting{DS/HLD.cpp}
\subsection{KD Tree}
\lstinputlisting{DS/KDTree.cpp}
\subsection{Lowest Common Ancestor}
\lstinputlisting{DS/LCA.cpp}
\subsection{Link Cut Tree}
\lstinputlisting{DS/link_cut_tree.cpp}
\subsection{PST}
\lstinputlisting{DS/PST.cpp}
\subsection{Rbst}
\lstinputlisting{DS/Rbst.cpp}
\subsection{pbds}
\lstinputlisting{DS/pbds.cpp}

\section{Flow}
\subsection{CostFlow}
\lstinputlisting{Flow/CostFlow.cpp}
\subsection{MaxFlow}
\lstinputlisting{Flow/Dinic.cpp}
\subsection{KM matching}
\lstinputlisting{Flow/KM_matching.cpp}
\subsection{Matching}
\lstinputlisting{Flow/Matching.cpp}

\section{Geometry}
\subsection{2D Geometry}
\lstinputlisting{Geometry/basic2D.cpp}
\subsection{3D ConvexHull}
\lstinputlisting{Geometry/ConvexHull3D.cpp}
\subsection{Half plane intersection}
\lstinputlisting{Geometry/Half_plane_intersection.cpp}


\section{Graph}
\subsection{2-SAT}
\lstinputlisting{Graph/2-sat.cpp}
\subsection{BCC}
\lstinputlisting{Graph/Biconnected_component.cpp}
\subsection{Bridge}
\lstinputlisting{Graph/Bridge.cpp}
\subsection{General Matching}
\lstinputlisting{Graph/Blossom.cpp}
\subsection{CentroidDecomposition}
\lstinputlisting{Graph/CentroidDecomposition.cpp}
\subsection{Diameter}
\lstinputlisting{Graph/diameter.cpp}
\subsection{DirectedGraphMinCycle}
\lstinputlisting{Graph/DirectedGraphMinCycle.cpp}
\subsection{General Weighted Matching}
\lstinputlisting{Graph/GeneralWeightedMatching.cpp}
\subsection{Graph Sequence Test}
\lstinputlisting{Graph/graph_sequence_test.cpp}
\subsection{maximal cliques}
\lstinputlisting{Graph/maximal_cliques.cpp}
\subsection{MinMeanCycle}
\lstinputlisting{Graph/MinMeanCycle.cpp}
\subsection{Prufer code}
\lstinputlisting{Graph/Prufer_code.cpp}
\subsection{SPFA}
\lstinputlisting{Graph/SPFA.cpp}
\subsection{Virtual Tree}
\lstinputlisting{Graph/VirtualTree.cpp}

\section{String}
\subsection{AC automaton}
\lstinputlisting{String/ac_automaton.cpp}
\subsection{KMP}
\lstinputlisting{String/kmp.cpp}
\subsection{Manacher}
\lstinputlisting{String/manacher.cpp}
\subsection{Suffix Array}
\lstinputlisting{String/suffix_array.cpp}
\subsection{Suffix Automaton}
\lstinputlisting{String/suffix_automaton.cpp}

\section{Formulas}
%\documentclass[a4paper,10pt,oneside]{article}
%\usepackage{xeCJK}
%\setCJKmainfont{微軟正黑體}
%\begin{document}

\subsection{Pick's theorem}
For a polygon: \\
$A$: The area of the polygon \\
$B$: Boundary Point: a lattice point on the polygon (including vertices)
$I$: Interior Point: a lattice point in the polygon’s interior region
$$A= I + \frac{B}{2} - 1$$

\subsection{Graph Properties}
\begin{enumerate}\itemsep = -5pt
\item Euler's Formula $V-E+F=2$
\item For a planar graph, $F=E-V+n+1$, n is the numbers of components
\item For a planar graph, $E\leq 3V-6$ \\

For a connected graph $G$:
$I(G)$: the size of maximum independent set
$M(G)$: the size of maximum matching
$Cv(G)$: be the size of minimum vertex cover
$Ce(G)$: be the size of minimum edge cover
\item For any connected graph:
  \begin{enumerate}\itemsep = -3pt
  \item $I(G)+Cv(G)=|V|$
  \item $M(G)+Ce(G)=|V|$
  \end{enumerate}
\item For any bipartite:
  \begin{enumerate}\itemsep = -3pt
  \item $I(G)=Cv(G)$
  \item $M(G)=Ce(G)$
   \end{enumerate}
\end{enumerate}



\subsection{Number Theory}
\begin{enumerate}\itemsep = -3pt
  \item $g(m)=\sum_{d|m}f(d)\Leftrightarrow f(m)=\sum_{d|m}\mu (d) \times g(m/d)$
  \item $\phi(x), \mu(x)$ are Möbius inverse
  \item $\sum_{i=1}^n\sum_{j=1}^m [\gcd(i, j) = 1]=\sum \mu(d)\left \lfloor \frac{n}{d} \right \rfloor \left \lfloor \frac{m}{d} \right \rfloor$
  \item $\sum_{i=1}^n\sum_{j=1}^nlcm(i,j)=n\sum_{d|n} d \times \phi (d)$
\end{enumerate}

\subsection{Combinatorics}
\DeclareRobustCommand{\stirling}{\genfrac\{\}{0pt}{}}
\begin{enumerate}
\itemsep = -3pt
  \item Gray Code: $=n\oplus (n>>1)$
  \item Catalan Number: $$C_n =  \frac{1}{n + 1} \binom{2n}{n} = \frac{(2n)!}{n!(n + 1)!} = \prod_{k=2}^n \frac{n + k}{k}$$
  \item $\Gamma(n + 1) = n!$
  \item $n! \approx \sqrt{2\pi n}  \left( \frac{n}{e} \right)^n$
  \item Stirling number of second kind: the number of ways to partition a set of n elements into k nonempty subsets.
	\begin{enumerate}\itemsep = -2pt
		\item $\stirling{0}{0} = \stirling{n}{n}=1$
		\item $\stirling{n}{0} = 0$
		\item $\stirling{n}{k} = k\stirling{n - 1}{k} + \stirling{n - 1}{k - 1}$
	\end{enumerate}
  \item Bell numbers count the possible partitions of a set:
	\begin{enumerate}\itemsep = -2pt
		\item $B_0=1$
		\item $B_n=\sum_{k=0}^n\stirling{n}{k}$
		\item $B_{n+1}=\sum_{k=0}^{n} C_k^n B_k$
		\item $B_{p+n}\equiv B{_n}+B_{n+1} \mod p$, p prime
		\item $B_{p^m+n}\equiv mB{_n}+B_{n+1} \mod p$, p prime
		\item From $B_0: 1,1,2,5,15,52,\\203,877,4140,21147,115975$
	\end{enumerate}
\item Derangement
	\begin{enumerate}\itemsep = -2pt
		\item $D_n=n!(1-\frac{1}{1!}+\frac{1}{2!}-\frac{1}{3!}\ldots +(-1)^n\frac{1}{n!})$
		\item $D_n=(n-1)(D_{n-1}+D_{n-2})$
		\item From $D_0: 1,0,1,2,9,44,\\265,1854,14833,133496$
	\end{enumerate}
\item Binomial\ Equality
	\begin{enumerate}\itemsep = -2pt
	    \item $\sum_k \binom{r}{m + k} \binom{s}{n - k} = \binom{r + s}{m + n}$
         \item $\sum_k \binom{l}{m + k} \binom{s}{n + k} = \binom{l + s}{l -m + n}$		
         \item $\sum_k \binom{l}{m + k} \binom{s + k}{n}(-1)^k = (-1)^{l + m} \binom{s - m}{n - l}$
		\item $\sum_{k\leq l} \binom{l - k}{m} \binom{s}{k - n}(-1)^k = (-1)^{l + m} \binom{s - m - 1}{l - n - m}$
		\item $\sum_{0 \leq k \leq l} \binom{l - k}{m} \binom{q + k}{n} = \binom{l + q + 1}{m + n + 1}$
		\item $\binom{r}{k} = (-1)^k\binom{k - r - 1}{k}$
		\item $\binom{r}{m} \binom{m}{k} = \binom{r}{k} \binom{r - k}{m - k}$
		\item $\sum_{k\leq n} \binom{r + k}{k} = \binom{r + n + 1}{n}$
		\item $\sum_{0\leq k \leq n} \binom{k}{m} = \binom{n + 1}{m + 1}$
		\item $\sum_{k\leq m}\binom{m + r}{k}x^ky^k = \sum_{k\leq m}\binom{-r}{k}(-x)^k (x+y)^{m-k}$	
	\end{enumerate}
\end{enumerate}


\subsection{Sum of Powers}
\begin{enumerate}\itemsep = -3pt
	\item $a^b\%P=a^{b\% \varphi (p)+\varphi (p)},b\geq \varphi (p)$
	\item $1^3+2^3+3^3+\ldots +n^3=\frac{n^4}{4}+\frac{n^3}{2}+\frac{n^2}{4}$
	\item $1^4+2^4+3^4+\ldots +n^4=\frac{n^5}{5}+\frac{n^4}{2}+\frac{n^3}{3}-\frac{n}{30}$
	\item $1^5+2^5+3^5+\ldots +n^5=\frac{n^6}{6}+\frac{n^5}{2}+\frac{5n^4}{12}-\frac{n^2}{12}$
	\item $0^k+1^k+2^k+\ldots +n^k = P_k,P_k=\frac{(n+1)^{k+1}-\sum_{i=0}^{k-1}C_i^{k+1}P(i)}{k+1},P_0=n+1$
	\item $\sum_{k=0}^{m-1}k^n=\frac{1}{n+1}\sum_{k=0}^{n}C_k^{n+1}B_km^{n+1-k}$
	\item $\sum_{j=0}^{m}C_j^{m+1}B_j=0,B_0=1$
	\item 除了$B_1=-1/2$,剩下的奇數項都是$0$
	\item $B_2=1/6,B_4=-1/30,B_6=1/42,B_8=-1/30,B_{10}=5/66,B_{12}=-691/2730,B_{14}=7/6,B_{16}=-3617/510,B_{18}=43867/798,B_{20}=-174611/330,$
\end{enumerate}

\subsection{Burnside's lemma}
\begin{enumerate}\itemsep = -3pt
	\item $|X/G| = \frac{1}{|G|}\sum_{g \in G}|X^g|$
	\item $X^g=t^{c(g)}$
\end{enumerate}
\subsection{Count on a tree}
\begin{enumerate}\itemsep = -3pt
	\item Rooted tree: $s_{n+1}=\frac{1}{n}\sum_{i=1}^{n}(i\times a_i\times \sum_{j=1}^{\left \lfloor  n/i\right \rfloor} a_{n+1-i\times j})$
	\item Unrooted tree: 
	\begin{enumerate}\itemsep = -2pt
		\item Odd:$a_n-\sum_{i=1}^{n/2}a_ia_{n-i}$
		\item Even:$Odd+\frac{1}{2}a_{n/2}(a_{n/2}+1)$
	\end{enumerate}
	\item Spanning Tree
	\begin{enumerate}\itemsep = -2pt
		\item 完全圖$n^n-2$
		\item 一般圖(Kirchhoff's theorem)$M[i][i]=\text{deg}(V_i)$,$M[i][j]=-1$,if have $E(i,j)$,$0$ if no edge. delete any one row and col in $A$, $ans = \det(A)$
	\end{enumerate}
\end{enumerate}

%\end{document}


\section{Team Comments}
\begin{enumerate}
\item 前一個小時把題目看完
\item 一個題目不只要想,還要想解題時間
\item while (有題目) 寫 // 不管多長
\item 盡快 AC 覺得可以快速 AC 的題目
\item rareone0602: 盡量不要讓我碰細節多的題目,盡量讓我想需要想突破口的題目
\item 如果目前沒有可寫的題目,先有希望題目的 IO
\item 讀過的題目可以像 priority queue 一樣,先花一些時間把題目塞進 pq 就說是 k 題好了,當 pq size 少於 k 把新題目塞進 pq
\item 電腦閒置可以生 debug 的測資
\end{enumerate}

\subsection{The Who-have-read Table}
\begin{tabular}{| l | c | c | r |}
\hline
   & rar & jjj & 0w1 \\ \hline
pA &     &     &     \\ \hline
pB &     &     &     \\ \hline
pC &     &     &     \\ \hline
pD &     &     &     \\ \hline
pE &     &     &     \\ \hline
pF &     &     &     \\ \hline
pG &     &     &     \\ \hline
pH &     &     &     \\ \hline
pI &     &     &     \\ \hline
pJ &     &     &     \\ \hline
pK &     &     &     \\ \hline
pL &     &     &     \\ \hline
\end{tabular}